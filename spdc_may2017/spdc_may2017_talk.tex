\documentclass{beamer}
\usetheme{Boadilla}
\usecolortheme{beaver}

\usepackage{color,colortbl}
\usepackage{setspace}
\usepackage{bbm}

\definecolor{maroonish}{RGB}{194, 27, 27}
\definecolor{lightgray}{gray}{0.85}
\definecolor{lightergray}{gray}{0.9}

\setbeamertemplate{bibliography item}[triangle]
\setbeamertemplate{caption}[numbered]

\setbeamercolor{item projected}{bg=darkred}
\setbeamercolor{itemize subitem}{fg=darkred}
\setbeamercolor{caption name}{fg=darkred}
\setbeamertemplate{itemize subitem}[triangle]

\title[Social Networks]{Social Networks: A Fast Tour From People to Groups!}
\author[Matt Hoover]{Matthew A. Hoover}
\date[]{02 May 2017}

\begin{document}
\frame{\titlepage}
% 0. Personal introduction
%     * Who I am
%     * What I've done
% 1. Introduction
%     * Background
%     * Outline talk
% 2. Individuals
%     * Basic unit of analysis for econometrics or ML
%     * Miss relational information
%     * What that provides
%     * Introduce:
%         * Node
% 3. What is a network
%     * Network creation
%     * Snapshot in time
%     * Varies based on tie
%         * Face-to-face network
%         * Online network
%     * One- and two-mode
%     * Complete versus personal
%     * Introduce:
%         * Edge
% 4. Basic network
%     * Dyad
%     * Introduce:
%         * Degree
%         * Direction
%     * Triad
%     * Introduce:
%         * Triad count
%         * Isolates
% 5. Larger network
%     * More than a triad
%     * Introduce:
%         * Centrality
%         * Sub-graphs
%     * How to relate to individual analyses
%         * Summaries
%         * Hierarchies
% 6. Community structure
%     * Network position
%     * What does this add to analyses
%     * Introduce:
%         * Block model
%         * Community structure
% 7. Network modeling
%     * Using network as the basis, not an individual
%     * Introduce:
%         * ERGM
%         * SAOM
% 8. Visualization
%     * Provide a 'summary view' on the network
%     * Helpful for qualitative analysis on its own
%     * Helps contextualize quantitative results
% 9. Building your own network
%     * Wealth of ability to build your own networks for analysis
%     * Most APIs will provide users; ties generated by data type
%     * Twitter; ties could be..
%         * Re-tweets
%         * Likes
%         * Followers
%         * Time of tweet
%         * Topic
%     * Untappd; ties could be...
%         * Same beer checked-in
%         * Location
%         * Friends
%         * Badges
%         * Use of language

% 0
\frame {
	\frametitle{Welcome!}
	\begin{itemize}
		\item<1-> Data scientist at Gallup
		\item<1-> Ph.D. in public policy
        \begin{itemize}
            \item<1-> Dissertation on the effects of children's social networks on education in rural Afghanistan
            \item<1-> Research on how social networks affected individual decision-making
        \end{itemize}
		\item<1-> Previous life, 15+ years in international development along with additional work for a healthcare startup
	\end{itemize}
}

% 1a
\frame {
	\frametitle{What is social network analysis?}
	\begin{itemize}
		\item<1-> Understanding the structure, composition, and purpose of people's social networks, whether in-person or online
		\item<1-> It helps answers questions from ``how do my friends and acquaintances affect my behaviors'' to ``from whom can I seek support in a given situation''
		\item<1-> \textbf{Structure} identifies how ties connect people in certain ways -- are there mutual ties, triangles, cycles?
		\item<1-> \textbf{Composition} describes the characteristics of people that are connected -- are they the same gender, about the same age?
		\item<1-> \textbf{Purpose} of a particular network varies -- is it a support network, drug seeking\slash using network, professional network?
	\end{itemize}
}

% 1b
\frame {
    \frametitle{What we'll cover today}
    \begin{itemize}
        \item<1-> Build up the conception of a network from an individual
        \item<1-> Introduce network measures: degree, centrality, triangles, and isolates
        \item<1-> Discuss analyses: community structure, block modeling, ERGMs, SAOMs
        \item<1-> Visualize networks and how/why that's important
        \item<1-> DIY: Building a network
        \item<2-> \textbf{The point-of-view for this talk are human networks, so the scale is considerably smaller -- and more tractable -- than large computer or web networks, like Facebook or Twitter}
    \end{itemize}
}

% 2
\frame {
    \frametitle{Let's start small: The individual}
    \begin{itemize}
        \item<1-> The individual plays the key role in most econometric analyses
        \item<1-> However, misses relational information between people
        \begin{itemize}
            \item<1-> Relations can guide actions or behaviors (\textit{influence})
            \item<1-> Actions can determine relations (\textit{selection})
        \end{itemize}
        \item<1-> In networks, an individual is called a \textbf{node}
        \begin{itemize}
            \item<1-> Terminology borrowed from graph theory, as another representation of a network is a graph
            \item<1-> Each node can have a series of attributes: age, gender, beliefs, career
            \item<1-> Note, networks do not have to be of people only
        \end{itemize}
    \end{itemize}
}

% 3a
\frame {
	\frametitle{A simple question: What is a network?}
	\begin{itemize}
		\item<1-> `I know it when I see it': Simply put, it's a collection of entities (nodes) connected in some way (edges)
        \item<1-> That collection of entities -- depending on how they are connected -- do or do not make a network\ldots moreover, depending on the connection type, it forms different networks
        \item<2-> The size of a network can differ dramatically; for example\ldots
        \begin{itemize}
            \item<2-> Friendship ties of high schoolers \textit{in real life}
            \item<2-> Friendship ties of high schoolers \textit{on Instagram}
        \end{itemize}
        \item<3-> The type of network can differ too\ldots
        \begin{itemize}
            \item<3-> A \textit{one-mode} network is of a single entity type, e.g., dogs connected to other dogs through breeding
            \item<3-> A \textit{two-mode} network is of two separate entities, e.g., people connected to beers they drank
            \item<3-> A two-mode network can be `flattened` to create a person-to-person or beer-to-beer network
        \end{itemize}
	\end{itemize}
}

% 3b
\frame {
    \frametitle{A simple question: What is a network?}
    \begin{itemize}
        \item<1-> The type of network can differ too\ldots
        \begin{itemize}
            \item<1-> A \textit{one-mode} network is of a single entity type, e.g., people connected in some way to other people
            \item<1-> A \textit{two-mode} network is of two separate entities, e.g., people connected to beers they drank
            \item<1-> A two-mode network can be `flattened` to create a person-to-person or beer-to-beer network
        \end{itemize}
        \item<2-> Finally, the scope of the network can differ
        \begin{itemize}
            \item<2-> A \textit{complete} network looks at the whole network, e.g., doctors' referrals to other doctors in a hospital
            \item<2-> A \textit{personal} or \textit{ego-centric} network focuses on the constellation of nodes around an entity in particular, e.g., whom jazz musicians have sessioned with in the past
            \item<2-> Both networks have their advantages and disadvantages
        \end{itemize}
    \end{itemize}
}

% 4a
\frame {
	\frametitle{The seeds of a network: Dyads and triads}
	\begin{itemize}
		\item<1-> The start of a network is the connection of two nodes through a edge: A \textit{dyad}
		\item<1-> This already adds complexity -- is the network \textit{undirected} or \textit{directed}?
        \begin{itemize}
            \item<2-> An undirected network means there is no directionality in the tie -- the tie is the same for both nodes, e.g., two servers that are connected to one another
            \item<2-> A directed network means ties have a sender and a receiver and the connection flows one-way only, e.g., followers on Twitter
            \item<2-> Of course, in a directed network and edge can be bi-directional, which is generally seen as a `stronger` tie than a uni-directional connection
        \end{itemize}
        \item<3-> A core building block in networks are \textit{triads} or the grouping of three nodes together in some way.
    \end{itemize}
}

% 4b
\frame {
    \frametitle{The seeds of a network: Dyads and triads}
    \begin{itemize}
        \item<1-> A triad, even though only three nodes, is already an interesting structure. It can exhibit hierarchy, closeness, or little relation between nodes
        \item<1-> With many nodes in a network, one technique to use is a \textit{triad count}, which countes the number of triads for the various possible formations
        \item<1-> Triads, depending on how they are formed and interconnect, help to determine the more advanced structure of a network, including cliques and sub-graphs
        \item<2-> An outgrowth of triads are \textit{isolates}, which are nodes in a network that are not connected to any other node
        \item<2-> Isolates often exhibit unique behavior or have attributes that differ from others within the network
        \item<2-> A network itself can be composed of many isolates, which is also interesting -- depending on the purpose of the network, this could be expected or potentially problematic
    \end{itemize}
}

% 5a
\frame {
	\frametitle{The total network}
	\begin{itemize}
		\item<1-> From triads, the network builds into more complex structures that can be broken down to isolates, dyads, and triads, but is much, much more
        \item<1-> At this point, network measures and statistics become important
        \item<1-> The most basic is \textit{density}, a measure on the network itself -- of all possible connections, how many are present?
        \item<1-> At the node level, there are measures of \textit{centrality}:
        \begin{itemize}
            \item<1-> \textit{Degree} (\textit{in-degree} and \textit{out-degree} for directed networks): The number of ties for each node in a network; in-degree\slash out-degree centrality can be called popularity and activity, respectively
            \item<1-> \textit{Betweenness}: A measure of position of a person -- do they sit `between' others or not; nodes will high betweenness centrality are `bridges' to other parts of the network
            \item<1-> \textit{Eigenvector}: Measures the connections' connections, that is, nodes have higher values if their connections are well-connected and those connections are well-connected and so on
        \end{itemize}
		\item<1-> Let's take a minute or two to look at a network and then take a look at some of these network summary measures
	\end{itemize}
}

% 5b
\frame {
    \frametitle{A directed network}
    \begin{figure}
    \begin{center}
        \includegraphics[height=.65\textwidth]{/Users/mhoover/Dropbox/git/presentations/spdc_may2017/network_FINAL}
        \label{network}
        \end{center}
    \end{figure}
}

% 5c
\frame {
    \frametitle{Network summary statistics}
    \begin{table}[ht]
        \centering
        \caption{Directed Network Summary}
        \begin{tabular}{lc}
            \hline
            Measure & Value \\
            \hline
            Number of Nodes & 48 \\
            Number of Edges & 48 \\
            Number of Isolates & 4  \\
            Network Density & 0.021 \\
            Degree Centralization & 0.044 \\
            \hline
        \end{tabular}
        \label{vill.demos}
    \end{table}
    \begin{itemize}
        \item<1-> Think about the following as we look at the graph again:
        \begin{itemize}
            \item<1-> Which nodes have high degree centrality?
            \item<1-> What about betweenness centrality? Why?
            \item<1-> Thoughts on high eigenvector centrality?
        \end{itemize}
    \end{itemize}
}

% 5d
\frame {
    \frametitle{A directed network}
    \begin{figure}
    \begin{center}
        \includegraphics[height=.65\textwidth]{/Users/mhoover/Dropbox/git/presentations/spdc_may2017/network_lab_FINAL}
        \label{network_lab}
        \end{center}
    \end{figure}
}

% 5e
\frame {
    \frametitle{Relating network information back to the individual}
    \begin{itemize}
        \item<1-> Let's pause for a second: So these measures may be \textit{interesting}, but what can be done with them beyond descriptive statistics?
        \item<1-> There needs to be a way to relate network summaries -- and the statistics we will discuss next -- to the individual
        \item<1-> There are two choices in particular:
        \begin{itemize}
            \item<2-> The first is using node-level summaries as individual-level measures
            \item<2-> These measures can be used as covariates in modeling efforts (remember to normalize!)
            \item<3-> Second, if there are individuals from multiple networks, then think about a hierarchical model, if the outcome is amenable
            \item<3-> Works well for analyses utilizing personal networks or similar complete networks measured at the same time
        \end{itemize}
        \item<4-> \textbf{Network analysis has suffered because people often can't figure out what to do with them in practice}
    \end{itemize}
}

% 6a
\frame {
    \frametitle{Moving from descriptives: Identifying structure in a network}
    \begin{itemize}
        \item<1-> Beyond network descriptive statistics and visualizations, we can use algorithms to determine underlying structure in networks
        \item<1-> The value of structure is it helps to sub-divide a network into smaller, (more) cohesive groups
        \item<2-> Some examples of community structure:
        \begin{itemize}
            \item<2-> Clients and personal relationships in commercial sex workers' lives
            \item<2-> Grade levels within a high school
            \item<2-> Political affiliation within Twitter
        \end{itemize}
        \item<3-> Of course, the algorithms only do the math -- the analyst needs to understand what the results mean
    \end{itemize}
}

% 6b
\frame {
    \frametitle{Community detection algorithms}
    \begin{itemize}
        \item<1-> Many variations; one of the more well-known algorithms is Girvan-Newman, which is a hierarchical method for detecting community structure
        \begin{enumerate}
            \item<1-> Calculate betweenness
            \item<1-> Remove edge with highest betweenness
            \item<1-> Repeat 1 and 2 until no edges are left
        \end{enumerate}
        \item<1-> Other algorithms exist, utilizing different rules for structuring
        \item<1-> Some will only work with undirected networks
        \item<1-> A community detection algorithm is not \textit{the answer} it is \textit{an answer} to help better understand what's happening in a network.
    \end{itemize}
}

% 6c
\frame{
    \frametitle{Girvan-Newman on our simple network}
    \begin{figure}[t]
    \begin{center}
        \includegraphics[width=.5\textwidth]{/Users/mhoover/Dropbox/git/presentations/spdc_may2017/network_lab_gn_FINAL}
        \includegraphics[width=.5\textwidth]{/Users/mhoover/Dropbox/git/presentations/spdc_may2017/gn_dendrogram_FINAL}
        \label{gn_structure}
    \end{center}
    \end{figure}
}

% 7a
\frame {
    \frametitle{Modeling networks, not individuals}
    \begin{itemize}
        \item<1-> Up to now, we've discussed operations that help identify network features for \textit{individual} analyses
        \item<1-> There are \textit{network-based} analyses that are possible now, with the advent of greater computing power
        \item<1-> Exponential random graph models (ERGMs)
        \begin{itemize}
            \item<1-> Cross-sectional
            \item<1-> Identifies structural and compositional elements of a network
            \begin{enumerate}
                \item<1-> Start with network representation
                \item<1-> Use MCMC to remove\slash add random edges
                \item<1-> Take `snapshot' of network
                \item<1-> Start with network representation; repeat
                \item<1-> Calculate how likely\slash unlikely given structural\slash compositional characteristics are in network representation, given all other network possibilities
            \end{enumerate}
        \end{itemize}
    \end{itemize}
}

% 7b
\frame {
    \frametitle{ERGM terms}
    \begin{table}[tbp]
    \begin{center}
    \tiny
    \begin{tabular}{>{\centering\arraybackslash}m{.75in} >{\centering\arraybackslash}m{.75in} >{\centering\arraybackslash}m{1.25in} >{\arraybackslash}m{1.25in}}
    \hline
    \textbf{Statistic} & \textbf{Visualization} & \textbf{Formula} & \textbf{Description} \\
    \hline
    \textsc{edges} & \includegraphics{edges_FINAL} & \[\sum_{i, j} y_{ij}\] & Sum of all ties in network \\
    \textsc{mutual} & \includegraphics{mutual_FINAL} & \[\sum_{i < j} y_{ij} y_{ji}\] & Sum of all reciprocated ties in network \\
    \textsc{twopath} & \includegraphics{twopath_FINAL} & \[\sum_{i \neq j \neq k} y_{ij} y_{jk}\] & Sum of all paths containing exactly one in-degree and one out-degree \\
    \textsc{gwidegree} & \includegraphics{gwidegree_FINAL} & \[\sum_{i=0}^n e^{-\alpha y_{+i}}\] & Indegree distribution, accounting for decrease in marginal utility of each additional nomination received \\
    \hline
    \end{tabular}
    \label{ergm.struc}
    \end{center}
    \end{table}
}

% 7c
\frame {
    \frametitle{ERGM terms}
    \begin{table}[tbp]
    \begin{center}
    \tiny
    \begin{tabular}{>{\centering\arraybackslash}m{.75in} >{\centering\arraybackslash}m{.75in} >{\centering\arraybackslash}m{1.25in} >{\arraybackslash}m{1.25in}}
    \hline
    \textbf{Statistic} & \textbf{Visualization} & \textbf{Formula} & \textbf{Description} \\
    \hline
    \textsc{gwodegree} & \includegraphics{gwodegree_FINAL} & \[\sum_{i=0}^n e^{-\alpha y_{i+}}\] & Outdegree distribution, accounting for decrease in marginal utility of each additional nomination sent \\
    \textsc{gwesp} & \includegraphics{gwesp_FINAL} &
    \tiny{\begin{multline*}
        e^{\theta_t} \sum_{i=1}^{n-1} \left\{ 1 - \left( 1 - e^{-\theta_t} \right)^i \right\} \\
        \sum_{i \neq j \neq k} y_{ij} y_{jk} y_{ik}
    \end{multline*}} & Transitive triplet distribution, accounting for decrease in marginal probability of closing triplet \\
    \textsc{ctriple} & \includegraphics{ctriple_FINAL} & \[\sum_{\substack{i \neq j \neq k \\ i < j, k}} y_{ij} y_{jk} y_{ki}\] & Sum of all cyclic triples in network \\
    \textsc{match} & \includegraphics{match_FINAL} & \[\sum_{i, j} y_{ij} \mathbbm{1} \{D_i = D_j\}\] & Sum of dyads matched on specified attribute \\
    \hline
    \end{tabular}
    \label{ergm.struc}
    \end{center}
    \end{table}
}
% 7d
\frame {
    \frametitle{ERGM estimates on our simple network}
    \begin{table}[ht]
    \centering
    \caption{Simple network ERGM parameters}
    \begin{tabular}{lccc}
    \hline
    Parameter & Estimate & Std. Error & $p$-value \\
    \hline
    edges & -15.73 & 3.928 & <0.001 \\
    mutual & 2.465 & 0.597 & <0.0011 \\
    twopath & -0.385 & 0.214 & 0.072 \\
    gwidegree & -2.790 & 1.136 & 0.014 \\
    gwodegree & 12.864 & 3.912 & 0.001 \\
    gwesp & 0.927 & 0.528 & 0.079 \\
    ctriple & 0.686 & 1.455 & 0.637 \\
    gender (match) & 3.437 & 1.011 & 0.001 \\
    \hline
    \end{tabular}
    \label{ergm}
    \end{table}
}

% 7e
\frame {
    \frametitle{Modeling networks over time}
    \begin{itemize}
        \item<1-> ERGMS can be thought of as a cross-sectional analysis; with networks, temporal analysis is possible as well
        \item<1-> Stochastic actor-oriented models (SAOMs) identify and measure the change in a network over time
        \item<1-> Help measure two processes within the network -- \textit{selection} and \textit{influence}
        \item<1-> Selection: Are connections chosen based on a shared attribute?
        \item<1-> Influence: Do connections induce behavior change
        \item<1-> Teen smoking: are friends chosen because others smoke (selection) or does smoking start because other friends are smoking (influence)
        \item<1-> Both processes can happen at the same time
    \end{itemize}
}

% 8a
\frame {
    \frametitle{Using visuals for qualitative analysis}
    \begin{itemize}
        \item<1-> Utilizing visuals can improve knowledge and understanding of a network
        \item<1-> Can utilize attributes on the network to help -- size, color, shape of nodes; can even use edge attributes (size, color, transparency)
        \item<1-> As with any visual, be careful that what's added doesn't ultimately distract from the message
    \end{itemize}
}

% 8b
\frame {
    \frametitle{A plain network}
    \begin{figure}
    \begin{center}
        \includegraphics[height=.65\textwidth]{/Users/mhoover/Dropbox/git/presentations/spdc_may2017/network_FINAL}
        \label{network1}
        \end{center}
    \end{figure}

}

% 8b
\frame {
    \frametitle{A network with shape}
    \begin{figure}
    \begin{center}
        \includegraphics[height=.65\textwidth]{/Users/mhoover/Dropbox/git/presentations/spdc_may2017/shape_network_FINAL}
        \label{network2}
        \end{center}
    \end{figure}

}

% 8d
\frame {
    \frametitle{A network with shape and color}
    \begin{figure}
    \begin{center}
        \includegraphics[height=.65\textwidth]{/Users/mhoover/Dropbox/git/presentations/spdc_may2017/shape_color_network_FINAL}
        \label{network3}
        \end{center}
    \end{figure}

}

% 9. Building your own network
%     * Wealth of ability to build your own networks for analysis
%     * Most APIs will provide users; ties generated by data type
%     * Twitter; ties could be..
%         * Re-tweets
%         * Likes
%         * Followers
%         * Time of tweet
%         * Topic
%     * Untappd; ties could be...
%         * Same beer checked-in
%         * Location
%         * Friends
%         * Badges
%         * Use of language
% 9a
\frame {
    \frametitle{Building networks}
    \begin{itemize}
        \item<1-> It is expensive and time-consuming to collect network data
        \item<1-> However, building a network doesn't need too difficult; with API access and some code, network construction is certainly possible
        \item<1-> Most APIs will provide some information on users (nodes) and based on the site, there are probably ways to identify connections (edges)
        \item<1-> Depending on the edges, this may create a one- or two-mode network
        \item<1-> Examples of one-mode connections:
        \begin{itemize}
            \item<1-> Twitter followers (one-mode)
            \item<1-> Friends on Instagram (one-mode)
            \item<1-> Hashtags used in tweets (two-mode)
            \item<1-> \textit{Beers checked-in on Untappd} (two-mode)
        \end{itemize}
        \item<1-> Let's take a look at building a one-mode network by collecting data from an API on a two-mode network and figure out some interesting things along the way
    \end{itemize}
}

% 9b
\frame {
    \frametitle{Creating a network from Untappd}
    \begin{itemize}
        \item<1-> \href{https://untappd.com/}{Untappd} is a mobile app that allows people to track the beers that they have or are drinking
        \item<1-> Its API provides access to user details as well as the beers that users check in (along with other information that we'll ignore for this exercise)
        \item<1-> So let's say I am interested to know not just \textit{who} my friends are (which I already have due to friend connections), but \textit{which of my friends have similar beer preferences}
        \item<1-> We could:
        \begin{itemize}
            \item<1-> Pull all the beers I have checked-in along will all of my friends
            \item<1-> For each friend, grab all the beers they've checked-in
            \item<1-> Create an edgelist of person-to-beer
            \item<1-> `Flatten' the two-mode network to a one-mode person-to-person (or, beer-to-beer) network, where the edges are the number of beers in common two nodes have checked in
        \end{itemize}
}

% 10
\frame {
	\frametitle{Thank you!}
	\begin{itemize}
		\item<1-> Thank you for your time!
		\item<1-> All materials are on my GitHub page: github.com/mhoover
        \begin{itemize}
            \item<1-> \texttt{presentations/spdc\_may2017}
            \item<1-> \texttt{ggnet}
            \item<1-> \texttt{untappd}
        \end{itemize}
		\item<1-> Questions?
	\end{itemize}
}

\end{document}
