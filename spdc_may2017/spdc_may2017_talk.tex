\documentclass{beamer}
\usetheme{Boadilla}
\usecolortheme{beaver}

\usepackage{color,colortbl}
\usepackage{setspace}

\definecolor{maroonish}{RGB}{194, 27, 27}
\definecolor{lightgray}{gray}{0.85}
\definecolor{lightergray}{gray}{0.9}

\setbeamertemplate{bibliography item}[triangle]
\setbeamertemplate{caption}[numbered]

\setbeamercolor{item projected}{bg=darkred}
\setbeamercolor{itemize subitem}{fg=darkred}
\setbeamertemplate{itemize subitem}[triangle]

\title[Social Networks]{Social Networks: A Fast Tour From People to Groups!}
\author[Matt Hoover]{Matthew A. Hoover}
\date[]{02 May 2017}

\begin{document}
\frame{\titlepage}
% 1. Introduction (1. Background, 2. Outline talk)
% 2. Individuals (1. Basic unit of analysis for econometrics or ML, 2. Miss relational information, 3. What that provides)
% 3. What is a network (1. Network creation, 2. Snapshot in time, 3. Varies based on tie (a. Face-to-face network, b. Online network), 4. One- and two-mode, 5. Complete versus personal, 6. Introduce: (a. Node, b. Edge))
% 4. Basic network
%     * Dyad
%     * Introduce:
%         * Degree
%         * Direction
%     * Triad
%     * Introduce:
%         * Triad count
%         * Isolates
% 5. Larger network
%     * More than a triad
%     * Introduce:
%         * Centrality
%         * Sub-graphs
%     * How to relate to individual analyses
%         * Summaries
%         * Hierarchies
% 6. Community structure
%     * Network position
%     * What does this add to analyses
%     * Introduce:
%         * Block model
%         * Community structure
% 7. Network modeling
%     * Using network as the basis, not an individual
%     * Introduce:
%         * ERGM
%         * SAOM
% 8. Visualization
%     * Provide a 'summary view' on the network
%     * Helpful for qualitative analysis on its own
%     * Helps contextualize quantitative results
% 9. Building your own network
%     * Wealth of ability to build your own networks for analysis
%     * Most APIs will provide users; ties generated by data type
%     * Twitter; ties could be..
%         * Re-tweets
%         * Likes
%         * Followers
%         * Time of tweet
%         * Topic
%     * Untappd; ties could be...
%         * Same beer checked-in
%         * Location
%         * Friends
%         * Badges
%         * Use of language

% 1
\frame {
	\frametitle{Welcome!}
	\begin{itemize}
		\item<1-> Data scientist at Gallup
		\item<1-> Ph.D. in public policy
        \begin{itemize}
            \item<1-> Dissertation on the effects of children's social networks on education in rural Afghanistan
            \item<1-> Research on how social networks affected individual decision-making
        \end{itemize}
		\item<1-> Previous life, 15+ years in international development along with additional work for a healthcare startup
	\end{itemize}
}

% 2
\frame {
	\frametitle{What is social network analysis?}
	\begin{itemize}
		\item<1-> Understanding the structure, composition, and purpose of people's social networks, whether in-person or online
		\item<1-> It helps answers questions from ``how do my friends and acquaintances affect my behaviors'' to ``from whom can I seek support in a given situation''
		\item<1-> \textbf{Structure} identifies how ties connect people in certain ways -- are there mutual ties, triangles, cycles?
		\item<1-> \textbf{Composition} describes the characteristics of people that are connected -- are they the same gender, about the same age?
		\item<1-> \textbf{Purpose} of a particular network varies -- is it a support network, drug seeking\slash using network, professional network?
	\end{itemize}
}

% 3
\frame {
    \frametitle{Let's start small: The individual}
    \begin{itemize}
        \item<1-> The individual plays the key role in most econometric analyses
    \end{itemize}
}

% 3
\frame {
	\frametitle{Types of analysis (thank goodness for computers!)}
	\begin{itemize}
		\item<1-> Early network analysis (circa 1920s) was done by hand on small networks (10-12 people)
		\item<1-> With computers, our reach is greatly expanded:
		\begin{itemize}
			\item<1-> Descriptive statistics
			\item<1-> Dyadic models
			\item<1-> Exponential random graph models
			\item<1-> Stochastic actor oriented models
		\end{itemize}
		\item<1-> Underneath it all though, visualization of networks has been an important analysis tool since the beginning
	\end{itemize}
}

% 4a
\frame {
	\frametitle{Visualization -- the problem}
	\begin{itemize}
		\item<1-> The problem with network visualization is quite simple: there are programs that do good visualization (\texttt{Gephi}, \texttt{Visone}) and there are programs that do good statistical work (\texttt{R}), but there were really none that did both (\texttt{Python} is starting to change this)
		\item<1-> This posed a problem to me, as I had 62 networks I needed to visualize for my dissertation and I didn't want to use a lot of different programs to accomplish my tasks (data management, analysis, visualization)
		\item<1-> Since each network can be thought of as a square matrix of 1's and 0's, I opted to figure out a solution inside \texttt{R} or with a program that \texttt{R} could plug into
		\item<1-> \texttt{Visone} had an \texttt{R} plug-in, but due to changes in \texttt{R} with release 3.0, that no longer worked
		\item<1-> I was stuck with visualizations in \texttt{R}, which left me with this:
	\end{itemize}
}

% 4b
\frame {
	\frametitle{Visualization -- the problem}
	\begin{figure}
		\begin{center}
		\includegraphics[height=.8\textwidth]{/Users/mhoover/Dropbox/pubs_presents/output/gplot_v01}
		\label{gplot1}
		\end{center}
	\end{figure}
}

% 5
\frame {
	\frametitle{(The start of) a solution}
	\begin{itemize}
		\item<1-> I was unhappy with a number of things:
		\begin{itemize}
			\item<1-> The color palette was jarring
			\item<1-> The resolution and plotting area was quite limited
			\item<1-> Controlling attributes (size\slash color of nodes) was difficult
			\item<1-> Most importantly, I could not ``set'' node coordinates to replicate a different network with the same actors
		\end{itemize}
		\item<1-> As I thought about solutions, I was using the \texttt{ggplot2} package in other work, so I thought I would try and create a \texttt{ggplot2}-like visualization for networks
		\item<1-> Let's look a bit more at the native and my \texttt{ggplot2}-inspired visualization
	\end{itemize}
}

% 6a
\frame {
	\frametitle{Native \texttt{R} plotting: \texttt{gplot}}
	\begin{itemize}
		\item<1-> Uses \texttt{R}'s base graphing features, layout, and language
		\item<1-> Limited color palette (8 colors), then it will recycle through colors
		\item<1-> Adding legends is difficult
		\item<1-> No way to capture\slash fix node coordinates
		\item<1-> However, very flexible in the format data can be fed to the function
		\item<1-> Let's look at it within \texttt{R}
	\end{itemize}
}

% 6b
\frame {
	\frametitle{\texttt{gplot}: Basic plot}
	\begin{figure}
	\begin{center}
		\includegraphics[height=.8\textwidth]{/Users/mhoover/Dropbox/pubs_presents/output/gplot_basic}
		\label{gplot2}
		\end{center}
	\end{figure}
}

% 6c
\frame {
	\frametitle{\texttt{gplot}: Colored nodes}
	\begin{figure}
	\begin{center}
		\includegraphics[height=.8\textwidth]{/Users/mhoover/Dropbox/pubs_presents/output/gplot_color}
		\label{gplot3}
		\end{center}
	\end{figure}
}

% 6d
\frame {
	\frametitle{\texttt{gplot}: Colored and shaped nodes}
	\begin{figure}
	\begin{center}
		\includegraphics[height=.8\textwidth]{/Users/mhoover/Dropbox/pubs_presents/output/gplot_color_shape}
		\label{gplot4}
		\end{center}
	\end{figure}
}

% 6e
\frame {
	\frametitle{\texttt{gplot}: With legend}
	\begin{figure}
	\begin{center}
		\includegraphics[height=.8\textwidth]{/Users/mhoover/Dropbox/pubs_presents/output/gplot_legend}
		\label{gplot5}
		\end{center}
	\end{figure}
}

% 7a
\frame {
	\frametitle{Plotting with \texttt{ggplot2}: \texttt{ggnet}}
	\begin{itemize}
		\item<1-> Utilizes \texttt{ggplot2} for setup and graphing
		\item<1-> Code builds on component pieces of \texttt{ggplot2}, such as lines, points, axes, and legends
		\item<1-> Can specify coordinates for nodes or use random layout
		\item<1-> Legends auto-generated
		\item<1-> Variety of color palettes and objects types can be used for attributes
		\item<1-> Requires a specific data object class (\texttt{network} class)
		\item<1-> Let's have a look within \texttt{R}
	\end{itemize}
}

% 7b
\frame {
	\frametitle{\texttt{ggnet}: Basic plot}
	\begin{figure}
	\begin{center}
		\includegraphics[height=.7\textwidth]{/Users/mhoover/Dropbox/pubs_presents/output/ggnet_basic}
		\label{ggnet1}
		\end{center}
	\end{figure}
}

% 7c
\frame {
	\frametitle{\texttt{ggnet}: Colored nodes}
	\begin{figure}
	\begin{center}
		\includegraphics[height=.7\textwidth]{/Users/mhoover/Dropbox/pubs_presents/output/ggnet_color}
		\label{ggnet2}
		\end{center}
	\end{figure}
}

% 7d
\frame {
	\frametitle{\texttt{ggnet}: Colored and shaped nodes}
	\begin{figure}
	\begin{center}
		\includegraphics[height=.7\textwidth]{/Users/mhoover/Dropbox/pubs_presents/output/ggnet_color_shape}
		\label{ggnet3}
		\end{center}
	\end{figure}
}

% 7e
\frame {
	\frametitle{\texttt{ggnet}: With legend}
	\begin{figure}
	\begin{center}
		\includegraphics[height=.7\textwidth]{/Users/mhoover/Dropbox/pubs_presents/output/ggnet_legend}
		\label{ggnet4}
		\end{center}
	\end{figure}
}

% 7f
\frame {
	\frametitle{\texttt{ggnet}: Same network, different ties}
	\begin{figure}
	\begin{center}
		\includegraphics[height=.5\textwidth]{/Users/mhoover/Dropbox/pubs_presents/output/ggnet_basic}
		\includegraphics[height=.5\textwidth]{/Users/mhoover/Dropbox/pubs_presents/output/ggnet_basic_friend}
		\label{ggnet5}
		\end{center}
	\end{figure}
}

% 7g
\frame {
	\frametitle{\texttt{ggnet}: Same network, different ties, node placement set}
	\begin{figure}
	\begin{center}
		\includegraphics[height=.5\textwidth]{/Users/mhoover/Dropbox/pubs_presents/output/ggnet_play_set}
		\includegraphics[height=.5\textwidth]{/Users/mhoover/Dropbox/pubs_presents/output/ggnet_friend_set}
		\label{ggnet6}
		\end{center}
	\end{figure}
}

% 7h
\frame {
	\frametitle{\texttt{ggnet}: Different color palette}
	\begin{figure}
	\begin{center}
		\includegraphics[height=.7\textwidth]{/Users/mhoover/Dropbox/pubs_presents/output/ggnet_palette}
		\label{ggnet7}
		\end{center}
	\end{figure}
}

% 7i
\frame {
	\frametitle{\texttt{ggnet}: Color gradient}
	\begin{figure}
	\begin{center}
		\includegraphics[height=.7\textwidth]{/Users/mhoover/Dropbox/pubs_presents/output/ggnet_gradient}
		\label{ggnet8}
		\end{center}
	\end{figure}
}

% 8
\frame {
	\frametitle{Comparing and contrasting \texttt{gplot} and \texttt{ggnet}}
	\begin{itemize}
		\item<1-> \texttt{gplot} has better flexibility for input; \texttt{ggnet} requires a specific data source
		\item<1-> \texttt{ggnet} has streamlined parameter entry
		\item<1-> \texttt{gplot} has a greater variety of attributes for nodes and edges
		\item<1-> \texttt{ggnet} has easier legend capabilities
		\item<1-> \texttt{gplot} can handle one- and two-mode networks; \texttt{ggnet} does not have the capability, currently
		\item<1-> \texttt{ggnet} allows for color gradient on continuous variables
		\item<1-> Both have extensive outputting capabilities for use outside of \texttt{R}
	\end{itemize}
}

% 9
\frame {
	\frametitle{Summary}
	\begin{itemize}
		\item<1-> Do it yourself! I had a need and while the solution isn't perfect, it was fun to put something together and keep working on it
		\item<1-> Next steps:
		\begin{itemize}
			\item<1-> Build out the two-mode features of \texttt{ggnet}
			\item<1-> Consolidate and create more efficient code
			\item<1-> Create an \texttt{R} package for submission to CRAN
		\end{itemize}
		\item<1-> Which function to use depends on need -- \texttt{gplot} for initial visualizations, \texttt{ggnet} for final visualizations
	\end{itemize}
}

% 10
\frame {
	\frametitle{Thank you!}
	\begin{itemize}
		\item<1-> Thank you for your time!
		\item<1-> Email: \href{mailto:matthew.a.hoover@gmail.com}{matthew.a.hoover@gmail.com}
		\item<1-> Questions?
	\end{itemize}
}

\end{document}
